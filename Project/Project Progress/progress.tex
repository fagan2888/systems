\documentclass[letterpaper]{article}
\title{Auto Scaling Online Learning}
\date{}
\usepackage{balance}  % to better equalize the last page
\usepackage{graphicx}
\usepackage{times}    % comment if you want LaTeX's default font
\usepackage{url}      % llt: nicely formatted URLs
\usepackage{graphicx}
\usepackage{tabularx}
\usepackage{float}
\usepackage{color}
\usepackage{url}
\usepackage[noend]{algpseudocode}
\usepackage{algorithm}
\usepackage{verbatim}
\usepackage{mathtools}
\usepackage{caption}
\usepackage{subcaption}
%\usepackage{amsmath}
\let\proof\relax
\let\endproof\relax
\usepackage{amsthm}
\usepackage{thmtools}
\usepackage{xspace}
\usepackage{multirow}
\setlength{\topmargin}{-.5in}
\setlength{\textheight}{9in}
\setlength{\oddsidemargin}{.125in}
\setlength{\textwidth}{6.25in}
\setlength{\parskip}{3pt}

\begin{document}
\author{CSE 550 Project Progress \\\\ Marco Tulio Correia Ribeiro, Shrainik Jain\\ 1323300, 1323338}
\maketitle

\section{Progress}
Our project relates to auto-scaling machine learning algorithms (more specifically, online
learning). For now, we will treat the machine learning aspect as a black box. Our plan is to
use Vowpal Wabbit\footnote{http://hunch.net/~vw/}, but in order to focus on the
systems side of things, we have decided to start with a dummy machine learning
server, that takes a determined time to do each operation. In this way, we can
focus on setting policies for auto-scaling and simulating different stream
speeds, without having to worry about all the ML complications.

The architecture we are using can we thought of as a
set of nodes acting as a learners and predictors. For further simplicity, we
assume only one node acting as a learner at a time. Other than this, there is a
load balancer (which can be a group of nodes) which handles incoming
requests from clients and
redirects them to learner and predictors as applicable. This load balancer is
also responsible for firing up new machines, or turning them off. All the
communication between the master node and the other nodes and clients is done
via Thrift\footnote{http://thrift.apache.org}. 
\\
The load balancer has information about all the running nodes. It also monitors
three criteria: cost (related to the number of machines running), mean
prediction time and learning rate. It makes decisions about starting or killing
machines based on a cost function that relates these three. Measuring the
learning rate seems not to be a trivial problem, and it is one that will require
more thinking in our part.

Further, the load balancer also implements a learning algorithm in order to
predict the future load (instead of waiting for bad prediction time sto happen),
given the load in the previous $t$ timestamps, in a manner similar to
\cite{jordan}. 

The clients communicate with the load balancer directly. As of now, we have 2 possible
models in mind for handling incoming requests:
\begin{itemize}
\item Master, upon recieving a request, updates the model for load prediction,
requests the learner or predictor as per the client request and relays the
response back to client. In this model, the master is a potential bottleneck.
\item Master, upon recieving a request from client, updates the model for load
prediction and returns a session and node to client. The client talks to the
node directly until the sesion is valid. The master can invalidate the session
if the node goes down is overly loaded. After session invalidation, the client
contacts master for a new session.
\end{itemize}

Irrespective of the model, the master keeps a list of valid nodes which are up
and keeps adding or removing nodes from this list based on load. Master also
sends regular heartbeat messages to nodes to check if some node has gone down,
and to check for the response time (in order to estimate the current load on the
node).

\bibliographystyle{plain}
\bibliography{references}

\end{document}
